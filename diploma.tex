% В этом шаблоне используется класс spbau-diploma. Его можно найти и, если требуется, 
% поправить в файле spbau-diploma.cls
\documentclass{spbau-diploma}
\begin{document}
% Год, город, название университета и факультета предопределены,
% но можно и поменять.
% Если англоязычная титульная страница не нужна, то ее можно просто удалить.
\filltitle{ru}{
    chair              = {Кафедра математических и информационных технологий},
    title              = {Анализ смесей близкородственных
бактериальных штаммов в метагеномных~сериях},
    % Здесь указывается тип работы. Возможные значения:
    %   coursework - Курсовая работа
    %   diploma - Диплом специалиста
    %   master - Диплом магистра
    %   bachelor - Диплом бакалавра
    type               = {master},
    position           = {студента},
    group              = 605,
    author             = {Аксешина Маргарита Дмитриевна},
    supervisorPosition = {},
    supervisor         = {Нурк С.\,Ю.},
    reviewerPosition   = {ст. преп.},
    reviewer           = {Кто-то?? А.\,И.},
    chairHeadPosition  = {д.\,ф.-м.\,н., профессор},
    chairHead          = {Омельченко А.\,В.},
    % university = {САНКТ-ПЕТЕРБУРГСКИЙ АКАДЕМИЧЕСКИЙ УНИВЕРСИТЕТ},
    % faculty = {Центр высшего образования},
    % city = {Санкт-Петербург},
    % year             = {2013}
}
\filltitle{en}{
    chair              = {Department of Mathematics and Information Technology},
    title              = {Analyzing mixtures of closely related strains in metagenomic series data},
    author             = {Margarita Akseshina},
    supervisorPosition = {},
    supervisor         = {Sergey Nurk},
    reviewerPosition   = {assistant},
    reviewer           = {Who??},
    chairHeadPosition  = {professor},
    chairHead          = {Alexander Omelchenko},
}


\maketitle


\tableofcontents


% ВВЕДЕНИЕ
\section*{Введение}

Метагеномика -- раздел геномики, изучающий ДНК, извлеченную и отсеквенированную непосредственно из среды без этапа изоляции и культивирования отдельного вида. Метагеномика позволяет исследовать, как отдельные микроорганизмы сосуществуют вместе, помогает понять функции и свойства сообществ. Например, средствами метагеномики можно отслеживать, как меняется и от чего зависит такая важная часть здоровья человека, как его микробиом \cite{HMP1, HMP2}. Кроме того метагеномика -- это часто единственный шанс изучить организмы, которые слишком сложно, долго, дорого или попросту невозможно изолировать от их среды.

Одной из самых распространенных и легко доступных технологий для получения метагеномных данных является чтение ДНК с помощью секвенирования нового поколения, результатом которого является набор коротких геномных фрагментов -- \textit{ридов}. Часто при этом в одном исследовании секвенируют сразу несколько связанных друг с другом образцов, собранных в разный момент времени \cite{time_series} или в разных местах~\cite{spacial_series_1, spacial_series_2}, получая таким образом \textit{метагеномные серии}.

Так как важные функциональные элементы генома (гены, регуляторные участки, повторы и т.д.) обычно не умещаются в один рид, чтобы проводить дальнейший анализ, эти короткие фрагменты с помощью метагеномных сборщиков \cite{IDBA-UD, MEGAHIT, MetaVelvet, RayMeta, MetaSpades} объединяют в более длинные последовательности -- \textit{контиги}. 

Чтобы оценить структуру популяции или восстановить геномы отдельных видов, используют методы \textit{биннинга}, -- то есть разделение метагеномных последовательности на группы так, чтобы в одной группе были последовательности только одного вида или, в крайнем случае, нескольких близкородственных. Как правило программы для биннинга \cite{CONCOCT, GroopM, MyCC, MetaBAT} кластеризуют последовательности, основываясь на их геномном составе (например, тетрануклеотидном спектре) и их покрытии в каждом из образцов. В тех алгоритмах, где используется сходство между профилями покрытия последовательностей в образцах, длина серии играет существенную роль в улучшении качества результатов биннинга.

Для выяснения таксономического состава образцов можно использовать инструменты, которые выравнивают последовательности на известные референсные геномы целиком \cite{GOTTCHA, CLARK, Kraken} или только на маркерные гены \cite{MetaPhlAn2, mOTU}. Однако такой метод не позволяет выявить те организмы, которые не присутствуют в базах.

Вышеперечисленные подходы работают в основном на уровне видов. Если в данных присутствуют близкородственные штаммы, сборка получается сильно фрагментированной, или контиги будут соответствовать только доминантному организму. Из-за схожести геномного состава программы биннинга будут группировать близкородственные штаммы вместе. 

Но многие исследования (такие, например, как секвенирование изолятов) показывают, что важные свойства микроорганизмов могут различаться внутри одного вида. Анализ штаммо-специфичных особенностей генома позволяет выяснять функциональные и патогенные свойства микроорганизмов (например, среди \textit{Escherichia coli} встречаются как штаммы-симбионты человеческого кишечника, так и патогенные \cite{patogen_ecoli} и карциногенные \cite{carcinogenic_ecoli}), проводить ассоциации генов с фенотипом хозяина (например, различные штаммы  \textit{Helicobacter pylori} ассоциируются с различным риском развития  рака желудка \cite{cancer_example}), отслеживать краткосрочные эволюционные события (например, появление резистентности к антибиотику \cite{antibitics_resistance}).

В некоторых методах риды выравниваются на референсные гены или геномы, чтобы выяснить вариации на уровне штаммов \textit{de novo}. Один из подходов -- брать в качестве гаплотипа в каждом образце нуклеотид, доминирующий в однонуклеотидной вариации \cite{ref_based2, ref_based1}. Но в одном образце может присутствовать несколько штаммов одного вида \cite{StrainEst, metasub, infant_gut}, а такой подход не позволяет восстановить смесь близкородственных штаммов, особенно те штаммы, которые не доминируют хотя бы в одном образце.

Существующие инструменты, анализирующие смеси близкородственных штаммов в образцах, основываются на анализе частот однонуклеотидных мутаций (SNVs) относительно выравнивания на геном или гены интересующего вида. В данном случае под частотой мутации понимается доля нуклеотидов, поддерживающих данную мутацию. Если мутация присутствует только в одном штамме смеси, её профиль частот в образцах будет соответствовать частоте этого штамма (Рис. \ref{snv_profile}, первая SNV). Однако если мутация общая для нескольких штаммов, они все уже будут вносить вклад в её профиль частот (Рис. \ref{snv_profile}, вторая SNV). Заметим, что такие соображения работают, только если фрагмент генома с рассматриваемой SNV встречается во всех штаммах и не попадает в повтор, иниче её профиль не будет отражать профиль частот штаммов (Рис. \ref{snv_profile}, третья SNV). В результате инструменты для анализа смесей штаммов относят каждую однонуклеотидную мутацию в одну из групп, соответствующую подмножеству штаммов, и выясняют частоты самих штаммов и их количество. ConStrains~\cite{Constrains} выравнивает риды на базу маркерных генов MetaPhlAn2 \cite{MetaPhlAn2} и дальше использует кластеризацию профилей частот и эвристические методы, пытаясь восстановить филогению штаммов. В StrainEst \cite{StrainEst} авторы тоже кластеризуют частоты SNVs, но получают векторы частот, выравнивая риды на все геномы интересующего вида, доступные в базе NCBI. Lineage \cite{Lineage} предлагает Байесовский метод анализа профилей частот SNVs (полученных как с использованием референса, так и без него) и, также как Constrains, выясняет филогенетическое родство между штаммами внутри вида.

\begin{figure}[t]
\centering
\includegraphics[width=0.9\textwidth]{pics/snv_profiles.png}
\caption{Подсчёт частот однонуклеотидныx мутаций (SNVs) для одного образца}
\label{snv_profile}
\end{figure}

DESMAN \cite{DESMAN} ищет в контигах однокопийные консервативные гены (те, которые должны присутствовать во всех штаммах рассматриваемого вида), и уже в них ищет SNVs. Затем он использует Байесовскую модель наподобие Lineage, но без учёта филогении, чтобы выяснить количество и частоты штаммов. После этого DESMAN выясняет, какому подмножеству штаммов принадлежит каждый из найденных неконсервативных генов (тех, которые могут отсутствовать в одном или нескольких штаммах). Таким образом, DESMAN --- это единственный из инструментов для анализа смесей штаммов в сериях \textit{de novo}, который выясняет не только профили встречаемости штаммов и однонуклеотидные мутации, но и информацию о геномном составе каждого их них.

Данная работа также посвящена выяснению состава каждого из смеси близкородственных штаммов в метагеномной серии, но с использованием помимо информации о профилях SNVs, еще и информации о структуре графа сборки. В графе сборке на основе графа де Брюйна \cite{de_Bruijn} вершины соответствуют фрагментам отсеквенированных геномов, а каждый геном представляет в графе связный путь. Таким образом граф даёт дополнительную информацию об устройстве геномов, которая потом теряется, если смотреть только на контиги. Кроме того свойства графа де Брюйна гарантируют нам, что фрагменты генома, соответствующие вершинам графа, не будут химерными (то есть ситуация, при которой разные части фрагмента принадлежат разным штаммам), в отличие от контигов.

На сегодняшний момент нам известен только один инструмент, использующий структуру графа сборки для анализа смесей близкородственных штаммов, --- это MaryGold \cite{MaryGold}, но он только ищет набор мест в графе, соответствующих вариациям между штаммами. Задачу, схожую по сути с поставленной, решают в транскриптомике при поиске изоформ гена \cite{flipflop2, other_flows, flipflop1}. Как в данной работе нам нужно определить, какому набору штаммов принадлежит геномный фрагмент, так при РНК-секвенировании нужно находить, из каких изоформ прочитался тот или иной экзон.

Таким образом целью данной работы является разработка пайплайна, который будет выявлять количество близкородственных штаммов в метагеномной серии, их процентное соотношение, а также для вершин графа сборки этих штаммов определять, какому подмножеству штаммов они принадлежат.


\section{Краткое описание подхода}

В данной работе мы будем анализировать графы сборки, полученные с помощью программы metaSpades \cite{MetaSpades}. Вершины в таком графе соответствуют фрагментам геномов, а направленное ребро соединяет вершину $A$ с вершиной $B$ в том и только том случае, если последние $k$ нуклеотидов вершины $A$ равны первым $k$ нуклеотидам вершины $B$. Мы подобрали параметры этапа симлификации алгоритма (то есть, упрощения структуры графа) таким образом, чтобы из графа не выкидывались вершины, соответствующие недоминантным штаммам, стараясь сохранить таким образом свойство о том, что геному каждого штамма в графе соответствует связный путь. Но при этом мы соединяли в одну вершины, соответствующие однонуклеотидным мутациям или коротким вставкам/удалениям, чтобы граф не был слишком большим.

В основе первого этапа нашего алгоритма лежит метод DESMAN. Мы используем его для того, чтобы найти количество штаммов и их процентное соотношение, а также определить, каким штаммам принадлежат вершины графа сборки, соответствующие длинным фрагментам геномов. Эта первоначальная классификация является отправной точкой нашего дальнейшего алгоритма, в котором мы пытаемся классифицировать остальные вершины, основываясь уже на структуре графа.



\section{Первоначальная классификация вершин с помощью DESMAN}

\subsection{Поиск SNVs}

Вначале нужно выравнять риды на референс, чтобы найти SNVs. В случае, когда уже известен геном хотя бы одного любого штамма интересующего вида, можно использовать его. Иначе, если референс неизвестен, можно использовать длинные контиги или вершины графа сборки с длинными последовательностями, которые по результатам биннинга принадлежат одному виду. 

Для выравнивания ридов мы использовали программу BWA-MEM \cite{bwa_mem}. Для поиска SNVs --- инструмент VarScan \cite{VarScan}; он позволяет не только находить позиции, в которых нуклеотиды отличаются от референса, но и отфильтровывать те, в которых вместо мутации присутствуют просто ошибки секвенирования. 

Отдельно хотим отметить, что на этом этапе можно оценить, присутствует ли в принципе в образцах смесь близкородственных штаммов интересующего вида. Так как данный вопрос выходит за основную тему работы, бодробнее в \ref{dominated_samples}.

\subsection{Фильтрация SNVs}

Чтобы правильно вывести из частот SNVs частоты штаммов, нужно брать только те позиции в геноме, которые не попадают на плохо покрытые участки, не принадлежат повторам и при этом присутствуют во всех штаммах рассматриваемого вида (см. пример на Рис. \ref{snv_profile}). Чтобы выполнить последние два условия, DESMAN ищет в контигах однокопийные консервативные гены и рассматривает мутации только в них. Авторы предлагают самостоятельно составлять базу таких генов для видов с большим количеством известных геномов (в своей статье они нашли 982 таких гена для E.Coli), а в случае, если доступных геномов мало, смотреть только на 36 генов, консервативных и однокопийных практически для всех прокариот. Но наши эксперименты (подробнее в \ref{infant_gut_section}) показали, что в случае, когда штаммы не слишком сильно отличаются друг от друга, в этих 36 генах случается мало мутаций, и из-за этого результаты для профилей частот штаммов оказываются неудовлетворительными. Поэтому мы ищем мутации во всём геноме и после этого фильтруем их. 

Мы считаем медиану покрытия всех найденных позиций с SNVs в каждом образце, предполагая, что она отражает реальное покрытие всех штаммов рассматриваемого вида в совокупности. Таким образом нам нужны позиции с SNVs, суммарное покрытие которых близко к медиане в каждом из образцов. Мы считаем евклидово расстояние между медианой и покрытием для каждой позиции и берём 3000 самых лучших (но это число можно менять в зависимости от условий, таких как степень родства штаммов и количество образцов).

\subsection{Количество штаммов и профили их частот}

Далее, имея вектора частот SNVs, мы можем вывести процентное соотношение с помощью одного из инструментов, упомянутых во введении. В данном случае мы используем DESMAN, как наиболее эффективный и строгий метод из всех доступных. Байесовская модель DESMAN предполагает знание количестве всех 4 нуклеотидов в рассматриваемых позициях --- эту информацию мы узнаём с помощью инструмента bam-readcount \cite{bam_readcount}. Другой инструмент, который мы применяли для этого --- это MIDAS \cite{midas}, но он подходит только в том случае, если в качестве референса берётся геном из его базы.

Для модели DESMAN количество штаммов в образцах является входным параметром алгоритма. Но авторы предлагают использовать график среднего постерионого отклонения для того, чтобы понять, начиная с какого количества штаммов отклонение перестаёт значительно уменьшаться. Наши эксперименты показали, что этот метод работает как для симулированных, так и для реального датасетов.

\subsection{Классификация вершин графа сборки}

В оригинальной статье авторы DESMAN после выяснения профилей частот штаммов предлагают использовать это знания для классификации по штаммам однокопийных неконсервативных генов на основе их покрытия, используя также Байесовский подход. Мы же в данной работе адаптировали данный алгоритм для геномных фрагментов, соответствующих длинным рёбрам в графе сборки, так как по сути для использования этого метода нам нужна только достоверная информация о покрытии фрагмента в каждом из образцов.

Так как при построении графа сборки metaSpades считает и сохраняет в индекс количество всех найденных k~-меров, мы можем использовать эту информацию для быстрой оценки нуклеотидного покрытия вершин по формуле $\frac{C_k r}{r - k + 1}$, где $C_k$ --- k-мерное покрытие, а $r$ -- длина ридов. Но в случае, если риды имеют большой разброс по длине, лучше искать покрытие с помощью честного выравнивания, --- у нас это реализовано с помощью утилиты bedtools genomecov \cite{bedtools}.


\section{Валидация и типичные проблемы в результатах DESMAN}

Для разработки и валидации алгоритма нам нужны были синтетические данные, которые обычно в метагеномных исследованиях получают путём симуляции ридов из набора референсных геномов. Но чтобы приблизить данные к реальным экспериментам (эффекты от ошибок секвенирования, распределение покрытия), мы в данной работе симулировали данные, получая их путём смешивания в разных пропорциях коротких ридов из секвенирований разных изолятов \textit{Escherichia coli} в рамках одного и того же геномного проекта (PRJNA239027 в базе NCBI BioProject). Авторы данной работы \cite{isolates} исследовали 6 изолятов \textit{E.coli}, присутствующие у 6 пациентов с разной степенью заражения крови. Кроме коротких ридов они провели секвенирование и сборку тех же изолятов с помощью PacBio технологии, и эти сборки мы использовали как референсные значения, к которым стремились. 

Мы сгенерировали случайным образом профили частот штаммов (СДЕЛАТЬ ТАБЛИЧКИ в приложении) таким образом, чтобы каждый штамм имел существенное покрытие хотя бы в одном образце. Мы рассматривали от 2 до 5 штаммов в 10 образцах, по 2 эксперимента на каждое количество штаммов с суммарным покрытием XXX в каждом образце. Покрытие одного изолята (ИМЯ) в изначальном эксперименте оказалась низким (УКАЗАТЬ СРЕДНЕЕ), и мы не брали его риды для симуляции.

Чтобы проверять результат получившейся классификации вершин, мы прикладывали рефенс к графу таким образом, чтобы он представлял в графе связный путь везде, где это возможно; для этого мы использовали внутренние инструменты лаборатории (что здесь дописать???). Таким образом для каждой вершины графа нам было известно, в каких штаммах и с каким числом копий она встречается.

Чтобы оценить результат количественно, мы рассматривали его как ответ на задачу бинарной классификации отдельно для каждого штамма (принадлежит вершина штамму или нет).

Наши эксперименты показали, что на тестовых данных DESMAN почти не даёт ложноположительных результатов на длинных вершинах. Исключения составляют вершины, соответствующие повторам, так как у них покрытие больше, чем предполагает модель. Гораздо чаще DESMAN делает ложноотрицательные выводы, и мы будем бороться в первую очередь с такими типами ошибок. Также мы будем дополнять ответ для коротких вершин.



\section{Уточнение классификации с использованием структуры графа}
\subsection{Однозначное продолжение пути в графе}
После первоначальной классификации с помощью DESMAN мы получили для каждого штамма набор длинных вершин, которые, как показало тестирование, с большой вероятностью ему принадлежат. Каждая из этих вершин должна лежать на пути в графе, образующем геном штамма. Поэтому если мы найдём, какие ещё вершины однозначно лежат на этом пути, мы сможем сказать, что с большой вероятностью они тоже принадлежат данному штамму. 

Таким образом в первую очередь мы можем проверить, нет ли среди длинных вершин каждого штамма такиx, что их исходящая степень равна 1. Тогда вершины, следующие за ними, тоже можно определить к этому же штамму (см Рис. \ref{one_continue}).

\begin{figure}[t]
\centering
\includegraphics[width=0.57\textwidth]{pics/one_continue.png}
\caption{У вершины $A$ только одно продолжение}
\label{one_continue}
\end{figure}

Но многие вершины имеют исходящую степень больше 1. Следовательно, через такую вершину существует несколько путей. В таком случае будем рассматривать один из двух вариантов. Вариант первый --- пути, выходящие из вершины, образуют \textit{пузырь}, то есть идут по разным вершинам, но потом вновь сходятся в одну (более строгое определение будет дано ниже). Тогда, если мы перескочим через этот пузырь, вершина на его конце тоже должна принадлежать рассматриваемому штамму, так как через неё однозначно проходят все возможные пути (см. Рис. \ref{bubbles_chain}). Кроме пузырей в графе существуют более сложные структуры, соответствующие повторяющимся участкам геномов, и пути через них уже не будут сходиться в одну вершину. Таким образом, если мы будем идти по графу вперёд, начиная от вершины, принадлежащей рассматриваемому штамму, то, пока нам на пути будут попадаться только пузыри, вершины между пузырями мы можем определять тоже как принадлежащие этому штамму. То же соображение работает, если мы будем идти от вершины назад по обратным рёбрам (см. Рис. \ref{bubbles_chain}).

\begin{figure}[t]
\centering
\includegraphics[width=0.8\textwidth]{pics/bubbles_chain.png}
\caption{Цепочка пузырей. Если оранжевая вершина принадлежит штамму, то жёлтые тоже должны ему принадлежать.}
\label{bubbles_chain}
\end{figure}

Чтобы искать пузыри и вершины между ними, мы воспользовались алгоритмом из статьи ССЫЛКА. Авторы данной работы вводят понятие \textit{суперпузыря} следующим образом: если упорядоченная пара вершин $(s,t)$ удовлетворяет следующим свойствам:
1) $t$ достижима из $s$;
2) набор вершин, достижимых из $s$ без прохождения через вершину $t$ равняется набору вершин, из которых достижима $t$ без прохождения через вершину $s$;
3) подграф, индуцированный набором $U$ вершин из предыдущего условия, --- ацикличен;
4) ни одна вершина из $U$, кроме $t$, не образует с $s$ пары, удовлетворяющей предыдущим условиям; то тогда множество вершин $U \cup \{s,t\}$ образуют \textit{суперпузырь}, а $s$ и $t$ --- это его начало и конец соответственно. Авторы предлагают алгоритм, основанный на стандартной топологической сортировке, который для каждой вершины за константное время в среднем определяет, является ли она началом \textit{суперпузыря} $s$, и если это так, находит его конец. 

Таким образом с помощью этого алгоритма мы можем найти цепочки пузырей такие, что конец одного пузыря совпадает с началом следующего. И тогда, если хотя бы один конец пузыря в этой цепочки принадлежит некоторому штамму, мы сможем сказать, что этот штамм проходит также через концы всех остальных пузырей в цепочке (см. Рис. ССЫЛКА). Заметим, что один пузырь может содержаться внутри другого пузыря, но классифицировать вершины пузыря мы будем дальше, поэтому концы таких <<внутренних>> пузырей на данном этапе мы рассматривать не будем. Также мы не будем рассматривать внутри пузырей однозначные продолжения, упомянутые выше.

\subsection{Пузыри в графе}

После того, как мы найдём цепочки пузырей и добавим к классификации их концов недостающие штаммы, мы можем сказать, какое подмножество штаммов должно проходить через каждый из пузырей. Тогда мы можем уточнить классификацию вершин внутри пузыря, основываясь на их покрытии. 

Каждый штамм в пузыре соответствует пути между его началом и концом. Мы можем перебрать все варианты путей и их соответствий штаммам, чтобы найти, какой набор путей наилучшим образом объясняет наблюдаемые покрытия вершин. Разумеется, количество возможных путей растёт экспоненциально с увеличением числа вершин в пузыре, но эксперименты показывают, что подавляющее число пузырей состоит из 2 вершин, и всего несколько превышают XXX вершин (ЗДЕСЬ НАДО ВСТАВИТЬ ГИСТОГРАММУ РАСПРЕДЕЛЕНИЯ КОЛИЧЕСТВА ВЕРШИН В РЕАЛЬНОМ И СИМУЛИРОВАННОМ ГРАФАХ И РАСПРЕДЕЛЕНИЕ КОЛИЧЕСТВА ВОЗМОЖНЫХ ПУТЕЙ).



ЗДЕСЬ ЕЩЁ НАПИСАТЬ, КАК ПОНИМАТЬ, КАКОЙ НАБОР ПУТЕЙ НАИЛУЧШИМ ОБРАЗОМ ОБЪЯСНЯЕТ ПОКРЫТИЕ

\subsection{Итоговый алгоритм}

ОФОРМИТЬ ВСЁ ВЫШЕСКАЗАННОЕ В ПСЕВДОКОД



\section{Результаты}
\subsection{Простые датасеты из смесей изолятов}
\subsection{Датасет Strain Mock}
\subsection{Реальный датасет} \label{infant_gut_section}






% У заключения нет номера главы
\section*{Заключение}
В данной работе 






\bibliographystyle{ugost2008ls}
\bibliography{diploma.bib}

%\appendix
\begin{appendices}

\section{Поиск смесей штаммов в образцах}\label{dominated_samples}

ПЕРЕПИСАТЬ

Опишем алгоритм поиска образцов с доминантными штаммами некоторого рассматриваемого вида. 

В каждом образце найдем однонуклеотидные мутации относительно референса рассматриваемого вида. Для каждой мутации считаем, сколько ридов поддерживает референс, а сколько – мутацию. Чтобы оставить только те мутации, которые не лежат в повторах и малопокрытых участках генома, возьмем только такие, чье суммарное покрытие близко к медиане покрытия этого вида в образце. Для каждой из выбранных мутаций подсчитаем частоту ридов, поддерживающих мутацию.

Каждая мутация специфична для какого-то одного штамма или некоторого множества штаммов. Если мы выбрали мутации с хорошим покрытием и не попадающие в повторы, то частота ридов, поддерживающих эту мутацию, должна примерно соответствовать сумме частот штаммов, для которых специфична эта мутация.

Тогда заметим, что если частота доминантного штамма – X, и X > 0.7, то мы должны наблюдать мутации с частотой порядка X и больше. При этом мутации, специфичные только для всех остальных штаммов, не могут давать частоты больше (1-X). Тогда в случае, когда в образце присутствует доминантный штамм, в распределении частот, соответствующих мутациям, мы должны наблюдать отсутствие значений в промежутке ((1-X); X).

Найдем величину промежутка эмпирически. Так как границы промежутка должны быть симметричны относительно значения 0.5, начнем с его середины и будем добавлять с обоих концов некоторую маленькую величину eps (например, 0.01) до тех пор, пока в промежуток попадает не больше 5\% от всех значений частот мутаций (те мутации, которые туда попали, считаем погрешностью). После того, как мы нашли предполагаемый промежуток, смотрим на его правую границу. Если она больше 0.7, считаем, что X > 0.7, и в образце есть доминантный штамм. Иначе – доминантного штамма нет.


\section{приложение два}
текст

\end{appendices}

\end{document}
